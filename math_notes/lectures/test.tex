\documentclass[11pt]{article}
\usepackage{amsmath, amssymb, amsthm}  % Essential math packages
\usepackage{graphicx}                   % For figures
\usepackage{hyperref}                   % Clickable links

\newtheorem{definition}{Definisjon}     % <-- added

\title{Exphil seminargruppe 1/6}
\author{Oliver Ekeberg}
\date{\today}

\begin{document}
\maketitle

\section{Argumenter og deres analyse}

\begin{flushleft}
    \begin{enumerate}
        \item Vitenskap er et systematisk søk etter viten eller kunnskap
        \begin{itemize}
            \item Kunnskap krever evidens eller begrunnelse
        \end{itemize}
        \item Viten er en sannhet med en velbegrunnet oppfatning
    \end{enumerate}
\end{flushleft}

\subsection{Hva er et argument?}

\begin{definition}
    \textbf{Et forsøk på å begrunne synspunkt eller oppfatning.}
\end{definition}

Et argument består av:
\begin{enumerate}
    \item Konklusjon
    \item Premiss
\end{enumerate}

Premissene kommer som støtte for konklusjonen.\\  
For eksempel:
\begin{itemize}
    \item Sokrates er et menneske
    \item Alle mennesker er dødelige
    \item Sokrates er dødelig
\end{itemize} 
\\
I dette eksemplet er premisset at alle mennesker er dødelige. \\
Dette er implisitt, fordi det ikke trengs en forklaring

\begin{definition}
    
\end{definition}

\end{document}
